\documentclass[twoside]{article} 
\usepackage{times}
\usepackage{amsmath,amssymb,amsfonts,verbatim,bm,graphicx}
\usepackage{enumerate,epstopdf,lscape,enumitem}
\usepackage[abbrvbib]{jmlr2e}

\begin{document}

\title{\textbf{Flexible Signal Denoising via Flexible Empirical Bayes
    Shrinkage}}

\author{\name Zhengrong Xing \email zhengrong@galton.uchicago.edu\\ 
\addr Department of Statistics \\
University of Chicago \\
Chicago, IL 60637, USA \\
\AND 
\name Matthew Stephens \email mstephens@uchicago.edu \\ 
\addr Department of Statistics and Department of Human Genetics \\
University of Chicago \\
Chicago, IL 60637, USA}
\editor{To be determined}
\date{}
\maketitle

\ShortHeadings{Flexible Denoising via Empirical Bayes Shrinkage}{Xing and
  Stephens}

\begin{abstract}
Signal denoising---also known as non-parametric regression---is often
performed through shrinkage estimation in a transformed (e.g.,
wavelet) domain; shrinkage in the transformed domain corresponds to
smoothing in the original domain. A key question in such applications
is how much to shrink, or, equivalently, how much to smooth. Empirical
Bayes shrinkage methods provide an attractive solution to this
problem; they use the data to estimate a distribution of underlying
``effects'', hence automatically select an appropriate amount of
shrinkage. However, most existing implementations of Empirical Bayes
shrinkage are less flexible than they could be---both in their
assumptions on the underlying distribution of effects, and in their
ability to handle heterskedasticity---which limits their signal
denoising applications. Here we address this by taking a particularly
flexible, stable and computationally convenient Empirical Bayes
shrinkage method, and we apply it to several signal denoising
problems. These applications include smoothing of Poisson data and
heteroskedastic Gaussian data. We show through empirical comparisons
that the results are competitive with other methods, including both
simple thresholding rules and purpose-built Empirical Bayes
procedures. Our methods are implemented in the R package {\tt smashr},
``SMoothing by Adaptive SHrinkage in R,'' available at
\url{https://www.github.com/stephenslab/smashr}.
\end{abstract}

{\bf Keywords:} Empirical Bayes, wavelets, non-parametric regression,
mean estimation, variance estimation

\section{Introduction}

Shrinkage and sparsity play a key role in many areas of modern
statistics, including high-dimensional regression
\citep{Tibshirani1996Regression}, covariance or precision matrix
estimation \citep{Bickel2008Covariance}, multiple testing
\citep{Efron2004} and signal denoising \citep{Donoho1994Ideal,
  donoho95}. One attractive way to achieve shrinkage and sparsity is
via Bayesian or Empirical Bayes (EB) methods
\citep[e.g.,][]{Efron2002Empirical, Johnstone2004Needles,
  Johnstone2005Empirical, Clyde2000Flexible,
  Daniels2001Shrinkage}. These methods are attractive because they can
adapt the amount of shrinkage to the available data. Specifically, by
learning from the data the distribution of the underlying ``effects''
that are being estimated, EB methods can appropriately adapt the
amount of shrinkage from data set to data set, and indeed from data
point to data point. For example, in settings where the effects are
sparse, but with a long tail of large effects, optimal accuracy is
achieved by strongly shrinking observations that lie near zero while
minimally shrinking the strongest signals \citep{polson2010shrink}.
This form of shrinkage can be achieved by appropriate EB methods.

One area where Bayesian methods for shrinkage have been found to be
particularly effective is in signal denoising
\citep{abramovich1998wavelet, Clyde2000Flexible,
  Johnstone2005Empirical}. Shrinkage plays a key role in signal
denoising, because signal denoising can be accurately and conveniently
achieved by shrinkage in a transformed (e.g., wavelet) domain
\citep{Donoho1994Ideal}. In empirical comparisons
\citep[e.g.,][]{Antoniadis2001Wavelet, Besbeas2004Comparative},
Bayesian methods often outperform alternatives such as simple
thresholding rules \citep{Coifman1995Translationinvariant,
  Donoho1994Ideal}. However, existing software implementations of
Bayesian and EB methods for this problem are limited; for example, the
{\tt EbayesThresh} package \citep{johnstone2005ebayesthresh} only
provides methods for estimating Gaussian means with constant
variance.

\bibliography{paper}

\end{document}
